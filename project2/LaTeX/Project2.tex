% Using template APS for projects
% Commented out non-useful code
% Skyter spurv med kanon

% ****** Start of file apssamp.tex ******
%
%   This file is part of the APS files in the REVTeX 4.1 distribution.
%   Version 4.1r of REVTeX, August 2010
%
%   Copyright (c) 2009, 2010 The American Physical Society.
%
%   See the REVTeX 4 README file for restrictions and more information.
%
% TeX'ing this file requires that you have AMS-LaTeX 2.0 installed
% as well as the rest of the prerequisites for REVTeX 4.1
%
% See the REVTeX 4 README file
% It also requires running BibTeX. The commands are as follows:
%
%  1)  latex apssamp.tex
%  2)  bibtex apssamp
%  3)  latex apssamp.tex
%  4)  latex apssamp.tex
%

\documentclass[%
 reprint,
 nobalancelastpage,
%superscriptaddress,
%groupedaddress,
%unsortedaddress,
%runinaddress,
%frontmatterverbose, 
%preprint,
%showpacs,preprintnumbers,
%nofootinbib,
%nobibnotes,
%bibnotes,
 amsmath,amssymb,
 aps,
%pra,
%prb,
%rmp,
%prstab,
%prstper,
%floatfix,
]{revtex4-1}

\usepackage{graphicx}% Include figure files
\usepackage{dcolumn}% Align table columns on decimal point
\usepackage{hyperref}% add hypertext capabilities
\usepackage{url}% url links
\usepackage{bm}% bold math
\usepackage{booktabs}% tables
\usepackage{listings}% codelisting
\usepackage{subcaption}% create subplots
\usepackage[labelformat=parens,labelsep=quad,skip=3pt]{caption}% caption plots
\usepackage{blindtext}% lorem ipsum...
%\usepackage[mathlines]{lineno}% Enable numbering of text and display math
%\linenumbers\relax % Commence numbering lines

%\usepackage[showframe,%Uncomment any one of the following lines to test 
%%scale=0.7, marginratio={1:1, 2:3}, ignoreall,% default settings
%%text={7in,10in},centering,
%%margin=1.5in,
%%total={6.5in,8.75in}, top=1.2in, left=0.9in, includefoot,
%%height=10in,a5paper,hmargin={3cm,0.8in},
%]{geometry}

\newcommand{\hbarm}{-\frac{\hbar^{2}}{2m}}
\newcommand{\ortwo}{\frac{1}{r^{2}}}
\newcommand{\ddr}{\frac{d}{dr}}
\newcommand{\ddrsq}{\frac{d^{2}}{dr^{2}}}
\newcommand{\ddrhosq}{\frac{d^{2}}{d\rho^{2}}}
\newcommand{\onehalf}{\frac{1}{2}}



\begin{document}

%\preprint{APS/123-QED}

\title{Project 2 - FYS3150}% Force line breaks with \\
\thanks{Computational Physics, autumn 2016, University of Oslo}%

\author{Andreas G. Lefdalsnes}
 % \altaffiliation[Also at ]{Physics Department, XYZ University.}%Lines break automatically or can be forced with \\

% \author{Second Author}%
%  \email{Second.Author@institution.edu}
\affiliation{%
 Student: University of Oslo, Department of Physics\\
 email-address: andregl@student.matnat.uio.no
}%

\author{Tellef Storebakken}
\affiliation{Student: University of Oslo, Department of Physics\\
 email-address: tellefs@student.matnat.uio.no}

% % \collaboration{MUSO Collaboration}%\noaffiliation

% % \author{Charlie Author}
% %  \homepage{http://www.Second.institution.edu/~Charlie.Author}
% % \affiliation{
% %  Second institution and/or address\\
% %  This line break forced% with \\
% % }%
% % \affiliation{
% %  Third institution, the second for Charlie Author
% % }%
% % \author{Delta Author}
% % \affiliation{%
% %  Authors' institution and/or address\\
% %  This line break forced with \textbackslash\textbackslash
% % }%

% % \collaboration{CLEO Collaboration}%\noaffiliation

\date{\today}% It is always \today, today,
             %  but any date may be explicitly specified

\begin{abstract}
In this project we solve the Schrodinger equation for two electrons in a 3D harmonic oscillator potential. We solve with and without electron repulsion, and compare the results. To accomplish this we apply a general method of discretizing the domain and reducing the problem to an eigenvalue equation. We thereafter apply Jacobi's rotation algorithm to obtain the eigenvalues of the matrix. We also apply the principles of unit testing by testing the algorithm for some simple problems with known solutions.

% % An article usually includes an abstract, a concise summary of the work
% % covered at length in the main body of the article. 
% % \begin{description}
% % \item[Usage]
% % Secondary publications and information retrieval purposes.
% % \item[PACS numbers]
% % May be entered using the \verb+\pacs{#1}+ command.
% % \item[Structure]
% % You may use the \texttt{description} environment to structure your abstract;
% % use the optional argument of the \verb+\item+ command to give the category of each item. 
% % \end{description}
\end{abstract}

% % \pacs{Valid PACS appear here}% PACS, the Physics and Astronomy
% %                              % Classification Scheme.
% % %\keywords{Suggested keywords}%Use showkeys class option if keyword
% %                               %display desired

\maketitle

\section{\label{sec:Int}Introduction}
In this project we aim to solve the Schrodinger equation for two electrons in a 3D harmonic oscillator potential. We will be solving with and without the repulsive Coulomb potential, and comparing the results. For the case of no repulsion we have an analytical expression for the energies, and this will be useful in determining the accuracy of our results. Assume spherical symmetry.

\section{\label{sec:The}Theory and methods}

\subsection{\label{sec:Rad}The radial equation}
We begin by studying the radial part of Schrodingers' equation for a single electron in a harmonic oscillator potential \footnote{All theory in this project adapted from FYS3150 Project 2 (Fall 2016) $\href{https://github.com/CompPhysics/ComputationalPhysics/tree/master/doc/Projects/2016/Project2}{<link>}$}.

\begin{equation}
	\hbarm	(\ortwo \ddr r^{2} - \frac{l(l+1)}{r^{2}}) R(r) + V(r)R(r) = E R(r)
\end{equation}

The potential $V(r) = \onehalf kr^{2}$ is the harmonic oscillator potential with $k = m\omega^{2}$ and E is the energy of the electron. $\omega$ is the oscillator frequency and the allowed energies are

\begin{equation}
	E_{nl} = \hbar \omega (2n+l+\frac{3}{2})
\end{equation}

Where the quantum number $n = 0,1,2..$ is the energy quantum number and $l = 0,1,2...$ is the orbital momentum quantum number. Introducing $R(r) = (1/r)u(r)$ our equation can be rewritten in terms of the second derivative $d^{2}/dr^{2}$:

\begin{equation}
	\hbarm \ddrsq u(r) + (V(r) + \frac{l(l+1)}{r^{2}}\hbarm)u(r) = Eu(r)
\end{equation}
\\
We introduce a dimensionless variable $\rho = (1/ \alpha)r$ where alpha is a constant of dimension length and obtain

\begin{equation}
	-\frac{\hbar^{2}}{2m\alpha^{2}} \ddrhosq u(\rho) + (V(\rho) + \frac{l(l+1)}{\rho^{2}}\frac{\hbar^{2}}{2m\alpha^{2}})u(\rho) = E u(\rho)
\end{equation}

In this project we will be interested in the case $l = 0$. Now since we are working in spherical coordinates, $r \in [0, \infty)$. Since we require $R(r)$ to go to zero at the boundaries, when we make the substituion $R(r) = (1/r)u(r) = (1/r)u(\alpha \rho)$ we obtain the boundary conditions for $u(\rho)$: $u(0) = u(\infty) = 0$.
\\ \\
We insert $V(\rho) = \onehalf k\alpha^{2}\rho^{2}$ and obtain

\begin{equation}
	-\frac{\hbar^{2}}{2m\alpha^{2}} \ddrhosq u(\rho) + \onehalf k\alpha^{2} \rho^{2}u(\rho) = E u(\rho)
\end{equation}

To obtain a simpler expression we multiply by $2m\alpha^{2} \rho^{2} /\hbar^{2}$ and fix $\alpha$ such that

\begin{equation}
	\frac{mk}{\hbar^{2}}\alpha^{4} = 1
\end{equation}

and define

\begin{equation}
	\lambda = \frac{2m\alpha^{2}}{\hbar^{2}}E
\end{equation}

so we can rewrite our equation as

\begin{equation}
	-\ddrhosq u(\rho) + \rho^{2} u(\rho) = \lambda u({\rho})
\end{equation}

To solve this equation we discretize the domain and define minimum and maximum values for $\rho$, $\rho_{min} = \rho_{0} = 0$ and $\rho_{max}$. $\rho_{max}$ cannot be chosen to be $\infty$ so we must take care to set it sufficiently large in order to obtain the correct solution. With $N$ mesh points let

\begin{equation}
	h = \frac{\rho_{max}-\rho_{0}}{N}
\end{equation}

and we obtain a discrete set of values for $\rho$, 

\begin{equation}
	\rho_{i} = \rho_{0} + ih \qquad i=0,1,2...,N
\end{equation}

Replacing the second order derivative by the 2nd order central difference we can write our equation as 

\begin{equation}
	-\frac{u_{i+1}+u_{i-1}-2u_{i}}{h^{2}} + \rho_{i}^{2}u_{i} = \lambda u_{i}
\end{equation}

Where $u_{i} = u(\rho_{i})$ is the discretized version of our function. We let $V_{i} = \rho_{i}^{2}$ and rewrite this as a matrix equation

\begin{equation}
	\bm{Au} = \lambda\bm{u}
\end{equation}

Since the endpoints are known we let $A$ be a matrix of dimension $(n-2)\cdot(n-2)$ and $u$ a vector of dimension (n-2).


\begin{widetext}
\begin{equation}
\bm{Au} =
\begin{pmatrix}
  \frac{2}{\hbar^{2}} + V_{1} & -\frac{1}{\hbar^{2}} & 0 & \cdots & \cdots & 0 \\
  -\frac{1}{\hbar^{2}} & \frac{2}{\hbar^{2}} + V_{2} &  -\frac{1}{\hbar^{2}} & 0 &\cdots & \cdots \\
  0 & -\frac{1}{\hbar^{2}} & \frac{2}{\hbar^{2}} + V_{3} & -\frac{1}{\hbar^{2}} & 0 & \cdots \\
  \vdots & \vdots & \vdots & \vdots & \vdots & \vdots \\
  0 & \cdots & \cdots & -\frac{1}{\hbar^{2}} & \frac{2}{\hbar^{2}} + V_{N-3} & -\frac{1}{\hbar^{2}} \\
  0 & \cdots & \cdots & \cdots & -\frac{1}{\hbar^{2}} & \frac{2}{\hbar^{2}} + V_{N-2}
\end{pmatrix}
\begin{pmatrix}
	u_{1} \\
	u_{2} \\
	\vdots \\
	\\
	\\
	u_{N-2}
\end{pmatrix}
=	\lambda
\begin{pmatrix}
	u_{1} \\
	u_{2} \\
	\vdots \\
	\\
	\\
	u_{N-2}
\end{pmatrix}
\end{equation}
\end{widetext}

where $u_{0} = u_{N-1} = 0$.

\subsection{Coulomb Interaction}
The single electron equation can be written as 


\section{RESULTS AND DISCUSSION}
First we checked how many mesh points $N$ we needed. We ran the program for a system where we knew the lowest eigenvalue and wanted the difference to be less than $10^{-4}$. When doing this for a known problem where the lowest eigenvalue should have been $\lambda_{theory} = 3$ we found that we needed $N=200$ mesh points to get this presicion.\\
  



%\tableofcontents

% \section{\label{sec:level1}First-level heading}

% This sample document demonstrates proper use of REV\TeX~4.1 (and
% \LaTeXe) in mansucripts prepared for submission to APS
% journals. Further information can be found in the REV\TeX~4.1
% documentation included in the distribution or available at
% \url{http://authors.aps.org/revtex4/}.

% When commands are referred to in this example file, they are always
% shown with their required arguments, using normal \TeX{} format. In
% this format, \verb+#1+, \verb+#2+, etc. stand for required
% author-supplied arguments to commands. For example, in
% \verb+\section{#1}+ the \verb+#1+ stands for the title text of the
% author's section heading, and in \verb+\title{#1}+ the \verb+#1+
% stands for the title text of the paper.

% Line breaks in section headings at all levels can be introduced using
% \textbackslash\textbackslash. A blank input line tells \TeX\ that the
% paragraph has ended. Note that top-level section headings are
% automatically uppercased. If a specific letter or word should appear in
% lowercase instead, you must escape it using \verb+\lowercase{#1}+ as
% in the word ``via'' above.

% \subsection{\label{sec:level2}Second-level heading: Formatting}

% This file may be formatted in either the \texttt{preprint} or
% \texttt{reprint} style. \texttt{reprint} format mimics final journal output. 
% Either format may be used for submission purposes. \texttt{letter} sized paper should
% be used when submitting to APS journals.

% \subsubsection{Wide text (A level-3 head)}
% The \texttt{widetext} environment will make the text the width of the
% full page, as on page~\pageref{eq:wideeq}. (Note the use the
% \verb+\pageref{#1}+ command to refer to the page number.) 
% \paragraph{Note (Fourth-level head is run in)}
% The width-changing commands only take effect in two-column formatting. 
% There is no effect if text is in a single column.

% \subsection{\label{sec:citeref}Citations and References}
% A citation in text uses the command \verb+\cite{#1}+ or
% \verb+\onlinecite{#1}+ and refers to an entry in the bibliography. 
% An entry in the bibliography is a reference to another document.

% \subsubsection{Citations}
% Because REV\TeX\ uses the \verb+natbib+ package of Patrick Daly, 
% the entire repertoire of commands in that package are available for your document;
% see the \verb+natbib+ documentation for further details. Please note that
% REV\TeX\ requires version 8.31a or later of \verb+natbib+.

% \paragraph{Syntax}
% The argument of \verb+\cite+ may be a single \emph{key}, 
% or may consist of a comma-separated list of keys.
% The citation \emph{key} may contain 
% letters, numbers, the dash (-) character, or the period (.) character. 
% New with natbib 8.3 is an extension to the syntax that allows for 
% a star (*) form and two optional arguments on the citation key itself.
% The syntax of the \verb+\cite+ command is thus (informally stated)
% \begin{quotation}\flushleft\leftskip1em
% \verb+\cite+ \verb+{+ \emph{key} \verb+}+, or\\
% \verb+\cite+ \verb+{+ \emph{optarg+key} \verb+}+, or\\
% \verb+\cite+ \verb+{+ \emph{optarg+key} \verb+,+ \emph{optarg+key}\ldots \verb+}+,
% \end{quotation}\noindent
% where \emph{optarg+key} signifies 
% \begin{quotation}\flushleft\leftskip1em
% \emph{key}, or\\
% \texttt{*}\emph{key}, or\\
% \texttt{[}\emph{pre}\texttt{]}\emph{key}, or\\
% \texttt{[}\emph{pre}\texttt{]}\texttt{[}\emph{post}\texttt{]}\emph{key}, or even\\
% \texttt{*}\texttt{[}\emph{pre}\texttt{]}\texttt{[}\emph{post}\texttt{]}\emph{key}.
% \end{quotation}\noindent
% where \emph{pre} and \emph{post} is whatever text you wish to place 
% at the beginning and end, respectively, of the bibliographic reference
% (see Ref.~[\onlinecite{witten2001}] and the two under Ref.~[\onlinecite{feyn54}]).
% (Keep in mind that no automatic space or punctuation is applied.)
% It is highly recommended that you put the entire \emph{pre} or \emph{post} portion 
% within its own set of braces, for example: 
% \verb+\cite+ \verb+{+ \texttt{[} \verb+{+\emph{text}\verb+}+\texttt{]}\emph{key}\verb+}+.
% The extra set of braces will keep \LaTeX\ out of trouble if your \emph{text} contains the comma (,) character.

% The star (*) modifier to the \emph{key} signifies that the reference is to be 
% merged with the previous reference into a single bibliographic entry, 
% a common idiom in APS and AIP articles (see below, Ref.~[\onlinecite{epr}]). 
% When references are merged in this way, they are separated by a semicolon instead of 
% the period (full stop) that would otherwise appear.

% \paragraph{Eliding repeated information}
% When a reference is merged, some of its fields may be elided: for example, 
% when the author matches that of the previous reference, it is omitted. 
% If both author and journal match, both are omitted.
% If the journal matches, but the author does not, the journal is replaced by \emph{ibid.},
% as exemplified by Ref.~[\onlinecite{epr}]. 
% These rules embody common editorial practice in APS and AIP journals and will only
% be in effect if the markup features of the APS and AIP Bib\TeX\ styles is employed.

% \paragraph{The options of the cite command itself}
% Please note that optional arguments to the \emph{key} change the reference in the bibliography, 
% not the citation in the body of the document. 
% For the latter, use the optional arguments of the \verb+\cite+ command itself:
% \verb+\cite+ \texttt{*}\allowbreak
% \texttt{[}\emph{pre-cite}\texttt{]}\allowbreak
% \texttt{[}\emph{post-cite}\texttt{]}\allowbreak
% \verb+{+\emph{key-list}\verb+}+.

\end{document}
%
% ****** End of file apssamp.tex ******